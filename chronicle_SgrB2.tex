%                                                                 aa.dem
% AA vers. 8.2, LaTeX class for Astronomy & Astrophysics
% demonstration file
%                                                       (c) EDP Sciences
%-----------------------------------------------------------------------
%
%\documentclass[referee]{aa} % for a referee version
%\documentclass[onecolumn]{aa} \SgrB\ % for a paper on 1 column  
%\documentclass[longauth]{aa} % for the long lists of affiliations 
%\documentclass[rnote]{aa} % for the research notes
%\documentclass[letter]{aa} % for the letters 
%\documentclass[bibyear]{aa} % if the references are not structured 
% according to the author-year natbib style

%
\documentclass{aa}
\usepackage{graphicx}
\usepackage{ebgaramond}
\usepackage{xspace} 
\usepackage{multirow}
\usepackage{dcolumn}
\newcolumntype{d}[1]{D{.}{.}{#1}}
\usepackage{hyperref}
\usepackage{xcolor}
\usepackage{soul}
% \usepackage{ulem}
\hypersetup{
    colorlinks,
    linkcolor={red!},
    citecolor={gray!},
    urlcolor={blue!},
    % draft,
    % hidelinks
}
%%%%%%%%%%%%%%%%%%%%%%%%%%%%%%%%%%%%%%%%
\usepackage{txfonts}
\usepackage{amsmath}
\usepackage{natbib}
\bibpunct{(}{)}{;}{a}{}{,} % to follow the A&A style
\usepackage{lscape}
\usepackage{multicol}
\usepackage[version=4]{mhchem}
\usepackage{booktabs}
% \usepackage{changes}
% \definechangesauthor[color=gray]{}
\usepackage{xcolor}
% \newcommand{\meng}[1]{\textcolor{red}{#1}}
\newcommand{\meng}[1]{#1}
\newcommand{\modc}[1]{\textbf{#1}}
\newcommand{\modb}[1]{\iffalse #1 \fi}
\newcommand{\moda}[1]{#1}


%%%%%%%%%%%%%%%%%%%%%%%%%%%%%%%%%%%%%%%%
\usepackage[detect-all,load=abbr)]{siunitx}
\sisetup{range-units=single}
\newcommand{\h}[1]{H{#1}{$\alpha$}\xspace}
\newcommand{\hii}{H{\sc ii }}
\newcommand{\kms}{ ${\rm km\ s^{-1}}$ }
\newcommand{\sgr}{Sgr\,B2 }
\newcommand{\uv}{\emph{uv}\xspace}
% \linespread{1.1}


% To add links in your PDF file, use the package "hyperref"
% with options according to your LaTeX or PDFLaTeX drivers.
%
\begin{document} 

\title{The physical and chemical structure of Sagittarius B2}
\titlerunning{UCH{\sc ii} regions in SgrB2}
\subtitle{VII. UCH{\sc ii} regions in SgrB2}
\author{F.~Meng\inst{1},
 {\'A}.~S{\'a}nchez-Monge\inst{1},
 P.~Schilke\inst{1},
 A.~Ginsburg\inst{2},
 A.~Schmiedeke\inst{3},
 A.~Schw{\"o}rer\inst{1},
 C.~DePree\inst{4},
 V.~S.~Veena\inst{1},
\and
Th.~M{\"o}ller\inst{1}
          }
\authorrunning{F. Meng et al.}
\institute{I.\ Physikalisches Institut, Universit\"at zu K\"oln, Z\"ulpicher Str.\ 77, D-50937 K\"oln, Germany\\
           \email{meng@ph1.uni-koeln.de}
           \and
           UFL, USA
           \and
           Max Planck Institute for Extraterrestrial Physics, Giessenbachstrasse 1, D-85748 Garching, Germany
           \and
           Agnes Scott College, 141 E. College Ave., Decatur, GA 30030, USA
           }

\date{Received ; accepted }
 
\abstract{The giant molecular cloud Sagittarius B2 (hereafter Sgr\,B2) is
the most massive region with ongoing high-mass star formation in the Galaxy. Dust cores and UC-\hii regions are spread all over Sgr\,B2. Giant dust and ionized bubbles fill the envelope of Sgr\,B2.} 
{We seek to characterize the association of UC-\hii regions and the 270 dust cores. } 
{We use the Very Large Array in its A, CnB and D configurations, and in the frequency bands C (4--8~GHz) and X (8--12~GHz) to observe the whole Sgr\,B2 complex. } 
{Dust cores and UC-\hii regions are spread all over Sgr\,B2. } 
{The envelope has star forming activities long ago. The star formation in SgrB2(M) and (N) is newer.}

\keywords{Stars: formation --
             Stars: massive --
             Radio continuum: ISM --
             Radio lines: ISM --
             ISM: clouds --
             ISM: individual objects: Sgr\,B2
             }

\maketitle

\section{Introduction} % (fold)
\label{sec:introduction}

  The giant molecular cloud Sagittarius B2 (SgrB2) is the most massive ($\sim 10^7\,M_{\odot}$) region with ongoing high-mass star formation in the Galaxy. SgrB2 has a higher density ($>10^5\rm\,cm^{-3}$) and dust temperature ($\sim$50--70\,K) compared to other star forming regions in the Galactic plane. Additionally, SgrB2 is located at a projected distance of only $\sim$100\,pc to the Galactic center. These features make SgrB2 an excellent case to study high-mass star formation in an extreme,  high-pressure environment. Such an environment resembles nearby starburst galaxies. Understanding the structure of the SgrB2 molecular cloud complex is necessary to comprehend the most massive star forming region in our Galaxy, which at the same time provides an unique opportunity to study in detail the nearest counterpart of the extreme environments that dominate star formation in the Universe. 

  In the central $\sim2\ \rm pc$ of SgrB2, there are the two well-known and studied hot cores Sgr\,B2(N) and Sgr\,B2(M) \citep[see e.g.,][]{Schmiedeke:2016aa,Sanchez-Monge:2017aa}, which contain at least 70 high-mass stars with spectral types from O5 to B0 \citep[see e.g.,][]{Gaume:1995aa,De-Pree:1998aa,De-Pree:2014aa}.  Surrounding the two hot cores, there is a larger envelope (hereafter \emph{the envelope}) with a radius of 20~pc that contains more than 99\% of the total mass of Sgr\,B2 \citep{Schmiedeke:2016aa}. The envelope has lower density $n_{\rm H} = 10^{3}\,\rm cm^{-3}$ and lower gas temperature $T\sim50\,\rm K$ compared to the two central hot cores of SgrB2. Alongside with the active high-mass star forming activities discovered in Sgr\,B2(N) and Sgr\,B2(M), hints of star formation happening in the envelope are also revealed. \citet{Ginsburg:2018aa}, with ALMA at 3~mm, revealed more than 271 high-mass protostellar cores distributed throughout the entire SgrB2, including the envelope. The luminosities of these dust cores suggest that they must contain objects with stellar masses larger than 8\,$M_\odot$. Due to the high extinction in infrared bands towards SgrB2 \citep[see][]{Meng:2019aa}, direct evidence of the existence of high-mass stars embedded in these cores are missing. However, since that high-mass stars ionize the neutral material surrounding them, the presence and properties of the associated \hii regions reflect the evolutionary stages of these dust cores \citep[see e.g.][]{Gonzalez-Aviles:2005aa,Breen:2010aa}. Additionally, since that the free-free emission from \hii regions may extend from cm to mm wavelength in spectral domain \citep[see e.g.][]{Sanchez-Monge:2013ab}, measuring the luminosities of the associated \hii regions can help us better constrain the luminosities of the dust cores. Therefore, to further characterize the evolutionary stages and physical properties of these dust cores, we aim to investigate the possible \hii regions associated with them. 

  The \hii regions in SgrB2 were targeted by various previous studies. \citep{Mehringer:1993aa} observed the entire SgrB2 with VLA in 20, 6, and 3.6\,cm bands and identified 15 \hii regions. The resolutions range from $\sim$20\arcsec to $\sim$3\arcsec from 20 to 3.6\,cm, which correspond to 0.8--0.12\,pc. The 15 \hii regions, except two unresolved ones, all have size $>2$\arcsec. Since that the 271 dust cores may contain high-mass stars newly formed \citep{Ginsburg:2018aa}, the associated \hii regions may be UCH{\sc ii} and HCH{\sc ii} regions, which typically have sizes from $\sim 0.01$\,pc to $\sim 0.1$\, pc \citep[see e.g.][]{Kurtz:2002aa,Gonzalez-Aviles:2005aa,Kurtz:2005aa,Breen:2010aa}. Such a resolution is not sufficient to resolve the UCH{\sc ii} and HCH{\sc ii} regions. \citet{Gaume:1990aa,Gaume:1995aa} observed SgrB2 at 7\,mm and 1.3\,cm and achieved resolutions of 0.065\arcsec and 0.25\arcsec, respectively. These high-resolution observations focus only on Sgr\,B2(N) and Sgr\,B2(M), which do not cover the envelope that the dust cores that are distributed throughout, especially SgrB2(DS) \citep[see e.g.][]{Ginsburg:2018aa,Meng:2019aa} that $\sim$80 of the cores reside. \citet{LaRosa:2000aa,Law:2008ab,Law:2008aa} have also observed the entire SgrB2 in cm wavelength but with resolutions not high enough to study UCH{\sc ii} and HCH{\sc ii} regions. 

  In this paper, we present Very Large Array (VLA) observations of the entire SgrB2 in the frequency regime 4--8~GHz, with configurations A, BnC and D. Thus, this study reaches resolutions of $\sim 0.05$. In \moda{Sect.}~2 we describe the observations as well as the data reduction process. \moda{Sect.}~3 shows the results. \moda{In Sect.~4 we discuss the results. Finally, we summarize this paper in Sect.~6.}

  \section{Observations and data reduction} % (fold)
  \label{sec:observations_and_data_reduction}

  We \moda{used} the VLA in \moda{its} A, CnB, and D configurations to observe the entire Sgr\,B2 complex in the frequency bands C (4--8~GHz) and X (8--12~GHz). The observations with the CnB and D configurations were described in \citep{Meng:2019aa}. The observations with the A configuration were conducted from October 1 to 12, 2016 (project 16B-031, PI: F.\  Meng). For both bands, we used 64 spectral windows with a bandwidth of 128~MHz each. Mosaic mode was used, with 10 and 18 pointings for each band, respectively. The primary beam of each pointing is 7.5\arcmin\ and 4.5\arcmin\ respectively. Quasar 3C286 was used as the flux and bandpass calibrator, the SED of which is $S_\nu = 5.059\pm0.021\ {\rm Jy} \times (S/8.435\ {\rm GHz})^{-0.46}$ from 0.5 to 50~GHz \citep{Perley:2013aa}.  The quasar J1820-2528, the flux of which is 1.3~Jy in the C and X bands, was used as phase calibrator. The calibration was done using the standard VLA pipelines provided by the NRAO\footnote{
        The National Radio Astronomy Observatory is a facility of the National Science Foundation operated under cooperative agreement by Associated Universities, Inc.}. 

  Calibration and imaging were done in Common Astronomy Software Applications (CASA) 4.7.2 \citep{McMullin:2007aa}. The details of data processing of data from CnB and D configurations are described in \citep{Meng:2019aa}, which results in two images in C and X band, respectively. The A configuration data were originally taken every 2\,s. To shorten the time of processing, we applied \texttt{timebin} in CASA to the measurement sets, which averaged the data taken within 10\,s into one data point. \moda{All the pointings of the mosaic in each band were primary beam corrected and the mosaic was imaged using the CASA task \texttt{tclean}.} With a robust factor of 0, the images of the C and X bands have \moda{synthesized beams} of 0.62\arcsec$\times$0.28\arcsec, with a position angle ${\rm (PA)}$ of $7.27^\circ$ and 0.36\arcsec$\times$0.15\arcsec (${\rm PA}=3.12^\circ$), respectively. The PA is defined positive north to east. To suppress the spatial filtering effect of the A configuration image, we applied the \texttt{feather} algorithm to combine the images from A configuration and the images from CnB and D configuration. The combined images have resolutions identical to that of A configuration images, while are sensitive to spatial scales up to $\sim240$\arcsec and $\sim145$\arcsec in C and X Band, respectively. The sensitivities of the observations are described in Sect.~\ref{sec:results}.

  \section{Results} % (fold)
  \label{sec:results}
  
  % section results (end)

  
  % section observations_and_data_reduction (end)

  % Single-dish telescopes have been used to map a large area around SgrB2 \citep[e.g.][]{de-Vicente:1997aa,de-Vicente:2000aa,Molinari:2011aa,Jones:2011aa}, achieving only coarse (20--40\arcsec) angular resolution. \citet{Martin-Pintado:1999aa} used the VLA to map an area of 3\arcmin$\times$3\arcmin\ in the ammonia inversion lines (3,3) and (4,4), with good angular resolution ($\sim$3\arcsec). Although covering only a small portion of the inner envelope, they find an interesting dense gas structure dominated by rings, arcs and filaments. The rings have sizes of 1--3~pc, thicknesses of 0.2--0.4~pc, contain warm gas ($\sim$40--70~K), and some of them have expansion velocities of $\sim$6--10~km~s$^{-1}$. The authors suggest that expanding bubbles, which produce these ammonia hot shells, are triggering new high-mass star formation in the envelope. Recent observations, with ALMA at 3~mm revealed more than 271 high-mass protostellar cores distributed throughout the envelope \citep{Ginsburg:2018aa}. These observations suggest that SgrB2 would resemble a ``swiss cheese'' with large holes dominating the large-scale structure and dense cores spread all over the envelope. Nevertheless, several questions still need to be answered. What are the physical properties of the 271 cores identified by \citet{Ginsburg:2018aa}? Are they starless or do already show signposts of star formation activity such as UC\hii regions or outflows?

  % Is the large scale ($\sim 20$\,pc) envelope of SgrB2 filled with ionized gas?  Are the expanding bubbles and arcs revealed by \citet{Martin-Pintado:1999aa} also present all over the envelope?

% section introduction (end)
\bibliographystyle{aa} % style aa.bst
\bibliography{ref}


\end{document}










