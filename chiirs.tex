%                                                                 aa.dem
% AA vers. 8.2, LaTeX class for Astronomy & Astrophysics
% demonstration file
%                                                       (c) EDP Sciences
%-----------------------------------------------------------------------
%
%\documentclass[referee]{aa} % for a referee version
%\documentclass[onecolumn]{aa} \SgrB\ % for a paper on 1 column  
%\documentclass[longauth]{aa} % for the long lists of affiliations 
%\documentclass[rnote]{aa} % for the research notes
%\documentclass[letter]{aa} % for the letters 
%\documentclass[bibyear]{aa} % if the references are not structured 
% according to the author-year natbib style

%
\documentclass{aa}
\usepackage{graphicx}
\usepackage{ebgaramond}
\usepackage{xspace} 
\usepackage{multirow}
\usepackage{dcolumn}
\newcolumntype{d}[1]{D{.}{.}{#1}}
\usepackage{hyperref}
\usepackage{xcolor}
\usepackage{soul}
% \usepackage{ulem}
\hypersetup{
    colorlinks,
    linkcolor={red!},
    citecolor={gray!},
    urlcolor={blue!},
    % draft,
    % hidelinks
}
%%%%%%%%%%%%%%%%%%%%%%%%%%%%%%%%%%%%%%%%
\usepackage{txfonts}
\usepackage{amsmath}
\usepackage{natbib}
\bibpunct{(}{)}{;}{a}{}{,} % to follow the A&A style
\usepackage{lscape}
\usepackage{multicol}
\usepackage[version=4]{mhchem}
\usepackage{booktabs}
% \usepackage{changes}
% \definechangesauthor[color=gray]{}
\usepackage{xcolor}
% \newcommand{\meng}[1]{\textcolor{red}{#1}}
\newcommand{\meng}[1]{#1}
\newcommand{\modc}[1]{\textbf{#1}}
\newcommand{\modb}[1]{\iffalse #1 \fi}
\newcommand{\moda}[1]{#1}


%%%%%%%%%%%%%%%%%%%%%%%%%%%%%%%%%%%%%%%%
\usepackage[detect-all,load=abbr)]{siunitx}
\sisetup{range-units=single}
\newcommand{\h}[1]{H{#1}{$\alpha$}\xspace}
\newcommand{\hii}{H{\sc ii }}
\newcommand{\kms}{ ${\rm km\ s^{-1}}$ }
\newcommand{\sgr}{Sgr\,B2 }
\newcommand{\uv}{\emph{uv}\xspace}
% \linespread{1.1}


% To add links in your PDF file, use the package "hyperref"
% with options according to your LaTeX or PDFLaTeX drivers.
%
\begin{document} 

\title{The physical and chemical structure of Sagittarius B2}
\titlerunning{UCH{\sc ii} regions in SgrB2}
\subtitle{VII. UCH{\sc ii} regions in SgrB2}
\author{F.~Meng\inst{1},
 {\'A}.~S{\'a}nchez-Monge\inst{1},
 P.~Schilke\inst{1},
 A.~Ginsburg\inst{2},
 A.~Schmiedeke\inst{3},
 A.~Schw{\"o}rer\inst{1},
 C.~DePree\inst{4},
 V.~S.~Veena\inst{1},
\and
Th.~M{\"o}ller\inst{1}
          }
\authorrunning{F. Meng et al.}
\institute{I.\ Physikalisches Institut, Universit\"at zu K\"oln, Z\"ulpicher Str.\ 77, D-50937 K\"oln, Germany\\
           \email{meng@ph1.uni-koeln.de}
           \and
           UFL, USA
           \and
           Max Planck Institute for Extraterrestrial Physics, Giessenbachstrasse 1, D-85748 Garching, Germany
           \and
           Agnes Scott College, 141 E. College Ave., Decatur, GA 30030, USA
           }

\date{Received ; accepted }
 
\abstract{The giant molecular cloud Sagittarius B2 (hereafter Sgr\,B2) is
the most massive region with ongoing high-mass star formation in the Galaxy. Dust cores and UC-\hii regions are spread all over Sgr\,B2. Giant dust and ionized bubbles fill the envelope of Sgr\,B2.} 
{We seek to characterize the association of UC-\hii regions and the 270 dust cores. } 
{We use the Very Large Array in its A, CnB and D configurations, and in the frequency bands C (4--8~GHz) and X (8--12~GHz) to observe the whole Sgr\,B2 complex. } 
{Dust cores and UC-\hii regions are spread all over Sgr\,B2. } 
{The envelope has star forming activities long ago. The star formation in SgrB2(M) and (N) is newer.}

\keywords{Stars: formation --
             Stars: massive --
             Radio continuum: ISM --
             Radio lines: ISM --
             ISM: clouds --
             ISM: individual objects: Sgr\,B2
             }

\maketitle

\section{Introduction} % (fold)
\label{sec:introduction}

  The giant molecular cloud Sagittarius B2 (SgrB2) is the most massive ($\sim 10^7\,M_{\odot}$) region with ongoing high-mass star formation in the Galaxy. SgrB2 has a higher density ($>10^5\rm\,cm^{-3}$) and dust temperature ($\sim$50--70\,K) compared to other star forming regions in the Galactic plane. Additionally, SgrB2 is located at a projected distance of only $\sim$100\,pc to the Galactic center. These features make SgrB2 an excellent case to study high-mass star formation in an extreme,  high-pressure environment. Such an environment resembles nearby starburst galaxies. Understanding the structure of the SgrB2 molecular cloud complex is necessary to comprehend the most massive star forming region in our Galaxy, which at the same time provides an unique opportunity to study in detail the nearest counterpart of the extreme environments that dominate star formation in the Universe. 

  In the central $\sim2\ \rm pc$ of SgrB2, there are the two well-known and studied hot cores Sgr\,B2(N) and Sgr\,B2(M) \citep[see e.g.,][]{Schmiedeke:2016aa,Sanchez-Monge:2017aa}, which contain at least 70 high-mass stars with spectral types from O5 to B0 \citep[see e.g.,][]{Gaume:1995aa,De-Pree:1998aa,De-Pree:2014aa}.  Surrounding the two hot cores, there is a larger envelope (hereafter \emph{the envelope}) with a radius of 20~pc that contains more than 99\% of the total mass of Sgr\,B2 \citep{Schmiedeke:2016aa}. The envelope has lower density $n_{\rm H} = 10^{3}\,\rm cm^{-3}$ and lower gas temperature $T\sim50\,\rm K$ compared to the two central hot cores of SgrB2. Alongside with the active high-mass star forming activities discovered in Sgr\,B2(N) and Sgr\,B2(M), hints of star formation happening in the envelope are also revealed. \citet{Ginsburg:2018aa}, with ALMA at 3~mm, revealed more than 271 high-mass protostellar cores distributed throughout the entire SgrB2, including the envelope. The luminosities of these dust cores suggest that they must contain objects with stellar masses larger than 8\,$M_\odot$. Due to the high extinction in infrared bands towards SgrB2 \citep[see][]{Meng:2019aa}, direct evidence of the existence of high-mass stars embedded in these cores are missing. However, since that high-mass stars ionize the neutral material surrounding them, the presence and properties of the associated \hii regions reflect the evolutionary stages of these dust cores \citep[see e.g.][]{Gonzalez-Aviles:2005aa,Breen:2010aa}. Additionally, since that the free-free emission from \hii regions may extend from cm to mm wavelength in spectral domain \citep[see e.g.][]{Sanchez-Monge:2013ab}, measuring the luminosities of the associated \hii regions can help us better constrain the luminosities of the dust cores. Therefore, to further characterize the evolutionary stages and physical properties of these dust cores, we aim to investigate the possible \hii regions associated with them. 

  The \hii regions in SgrB2 were targeted by various previous studies. \citep{Mehringer:1993aa} observed the entire SgrB2 with VLA in 20, 6, and 3.6\,cm bands and identified 15 \hii regions. The resolutions range from $\sim$20\arcsec to $\sim$3\arcsec from 20 to 3.6\,cm, which correspond to 0.8--0.12\,pc. The 15 \hii regions, except two unresolved ones, all have size $>2$\arcsec. Since that the 271 dust cores may contain high-mass stars newly formed \citep{Ginsburg:2018aa}, the associated \hii regions may be UCH{\sc ii} and HCH{\sc ii} regions, which typically have sizes from $\sim 0.01$\,pc to $\sim 0.1$\, pc \citep[see e.g.][]{Kurtz:2002aa,Gonzalez-Aviles:2005aa,Kurtz:2005aa,Breen:2010aa}. Such a resolution is not sufficient to resolve the UCH{\sc ii} and HCH{\sc ii} regions. \citet{Gaume:1990aa,Gaume:1995aa} observed observed Sgr\,B2(N) and Sgr\,B2(M) at 7\,mm and 1.3\,cm and achieved resolutions of 0.065\arcsec and 0.25\arcsec, respectively. \citet{Rolffs:2011aa} observed Sgr\,B2(N) and Sgr\,B2(M) in 40\,GHz with resolution of 0.1\arcsec. Unfortunately, these high-resolution observations do not cover the envelope that the dust cores that are distributed throughout, especially SgrB2(DS) \citep[see e.g.][]{Ginsburg:2018aa,Meng:2019aa} that $\sim$80 of the dust cores reside. Additionally, \citet{LaRosa:2000aa,Law:2008ab,Law:2008aa} have also observed the entire SgrB2 in cm wavelength but with resolutions not high enough to study UCH{\sc ii} and HCH{\sc ii} regions. 

  In this paper, we present Very Large Array (VLA) observations of the entire SgrB2 in the frequency regime 4--8~GHz, with configurations A, BnC and D. Thus, this study reaches resolutions of $\sim 0.05$. For comparison, we also include analysis of the 3\,mm image \citep{Ginsburg:2018aa} as well as the newly acquired \ce{SiO}\,(5--4) data, both of which were observed with ALMA (the Atacama Large Millimeter/submillimeter Array). In \moda{Sect.}~2 we describe the observations as well as the data reduction process. \moda{Sect.}~3 shows the results. \moda{In Sect.~4 we discuss the results. Finally, we summarize this paper in Sect.~6.}

\section{Observations and data reduction} % (fold)
\label{sec:observations_and_data_reduction}

  We \moda{used} the VLA in \moda{its} A, CnB, and D configurations to observe the entire Sgr\,B2 complex in the frequency bands C (4--8~GHz) and X (8--12~GHz). The observations with the CnB and D configurations were described in \citep{Meng:2019aa}. The observations with the A configuration were conducted from October 1 to 12, 2016 (project 16B-031, PI: F.\  Meng). For both bands, we used 64 spectral windows with a bandwidth of 128~MHz each. Mosaic mode was used, with 10 and 18 pointings for each band, respectively. The primary beam of each pointing is 7.5\arcmin\ and 4.5\arcmin\ respectively. Quasar 3C286 was used as the flux and bandpass calibrator, the SED of which is $S_\nu = 5.059\pm0.021\ {\rm Jy} \times (S/8.435\ {\rm GHz})^{-0.46}$ from 0.5 to 50~GHz \citep{Perley:2013aa}.  The quasar J1820-2528, the flux of which is 1.3~Jy in the C and X bands, was used as phase calibrator. The calibration was done using the standard VLA pipelines provided by the NRAO\footnote{
        The National Radio Astronomy Observatory is a facility of the National Science Foundation operated under cooperative agreement by Associated Universities, Inc.}. 

  Calibration and imaging were done in Common Astronomy Software Applications (CASA) 4.7.2 \citep{McMullin:2007aa}. The details of data processing of data from CnB and D configurations are described in \citep{Meng:2019aa}, which results in two images in C and X band, respectively. The A configuration data were originally taken every 2\,s. To shorten the time of processing, we applied \texttt{timebin} in CASA to the measurement sets, which averaged the data taken within 10\,s into one data point. \moda{All the pointings of the mosaic in each band were primary beam corrected and the mosaic was imaged using the CASA task \texttt{tclean}.} With a robust factor of 0, the images of the C and X bands have \moda{synthesized beams} of 0.62\arcsec$\times$0.28\arcsec, with a position angle ${\rm (PA)}$ of $7.27^\circ$ and 0.36\arcsec$\times$0.15\arcsec (${\rm PA}=3.12^\circ$), respectively. The PA is defined positive north to east. To suppress the spatial filtering effect of the A configuration image, we applied the \texttt{feather} algorithm to combine the images from A configuration and the images from CnB and D configuration. The combined images have resolutions identical to that of A configuration images, while are sensitive to spatial scales up to $\sim240$\arcsec and $\sim145$\arcsec in C and X Band, respectively. The sensitivities of the observations are described in Sect.~\ref{sec:results}. 

  Ancillary data include the 3\,mm continuum and \ce{SiO}\,(5--4) data. The 3\,mm continuum data covers frequency range from 89.5 too 103.3\,GHz. The image in 3\,mm has a resolution of 0.54\arcsec$\times$0.46\arcsec (${\rm PA}=68.31^\circ$). The observational details of the 3\,mm data are described by \citet{Ginsburg:2018aa}. The SiO~(5--4) emission was observed with ALMA (Project 18A-229, P.I. A. Ginsburg). For the details of the observation and data reduction, see Ginsburg et. al. (in prep.). The resolution is 0.35\arcsec $\times$ 0.24\arcsec, with P.A. of $-80^{\circ}$. The spectral resolution is 1.35~km~s$^{-1}$. The typical RMS of the image is 0.9~mJy/beam. The observation covers SgrB2(S) and the eastern part of SgrB2(DS). 


 \section{Results} % (fold)
 \label{sec:results}

  \begin{figure*}[ht]
          \begin{center}
          \includegraphics[width=0.45\textwidth]{./plot/abcdC_map.pdf}
          \includegraphics[width=0.45\textwidth]{./plot/abcdX_map.pdf}\\     
          \caption[Images in C and X Bands.]{Images in C and X Bands. Notable regions are marked.
           Beam sizes are marked on the lower left corner of each panel, in format $(\theta_{\rm maj}, \theta_{\rm min}, {\rm PA})$. 
           Each image has two profile cuts, in blue and red. The offset of each profile increase from left to right and from bottom to top. The profiles are below each image.
           }
          \label{f:abcdcxmap}
          \end{center}
  \end{figure*}

  In this section, we present the image of SgrB2 in 6\,cm and the UCH{\sc ii} regions identified in it. We also compared between these ionized cores with dust cores idetified in 96~GHz. Statistics of the properties of these UCH{\sc ii} regions are shown. 

  We show the images in C and X bands in Fig.~\ref{f:abcdcxmap}, in which the known large scale \hii regions `N', `M', `S', `AA', `DS', and `V' are denoted, following the nomenclature of \citet{Mehringer:1992aa,Mehringer:1993aa,Ginsburg:2018aa,Meng:2019aa}. Although that deep CLEAN and self calibration were preformed in data processing, the high dynamic range of the images ($\sim 10^3$) hinders the elimination of artifacts. As shown in Fig.~\ref{f:abcdcxmap}, the X band image is highly contaminated by the artifacts that from which the real emission is hard to distinguished. Therefore, we do not take the X band image into account in the following analysis. The C band image is called `6\,cm image' in the following analysis. Due to the non-homogeneity of the artifacts over the map, we do not give an RMS (root-mean-square) value to represent the noise value of the 6\,cm image, but evaluate the spatial varying RMS. The procedures of generation the RMS maps are described in Sect.~\ref{sec:rms_maps}. As shown in Fig.~\ref{f:rmsmap_a}, the RMS of the 6\,cm image varies from $\sim$0.02\,mJy/beam to $\sim$0.2\,mJy/beam. The RMS of the 3\,mm map, as shown in Fig.~\ref{f:rmsmap_b}, ranges from $\sim$0.1\,mJy/beam to $\sim$0.4\,mJy/beam. 

  \subsection{UCH{\sc ii} regions} % (fold)
  \label{sub:cores_in_vla}

  The identification of sources in the 6\,cm image was automatically conducted with `SExtractor' \citep{Bertin:1996aa}. The details of source identification are described in Sect.~\ref{sec:core_identification}. The sources are fitted with ellipses which have three parameters, half of major axis ($a$), half of minor axis ($b$), and position angle ($\vartheta$). The position angle ($\vartheta$) is defined as from east to north.  To select only candidates of UCH{\sc ii} regions , we constrain $a\leq$0.1\arcsec and $a\leq3b$. The former criterion is based on that the typical sizes of UCH{\sc ii} regions are less than 0.1\arcsec \citep[see e.g.][]{Kurtz:2002aa,Kurtz:2005aa}. The later criterion is to avoid mis-identifying of segments of filaments as UCH{\sc ii} regions. 
  %In this paper, we do not specifically stress on HCH{\sc ii} but only call all the identified UCH{\sc ii} regions and HCH{\sc ii} regions as `UCH{\sc ii} regions'.

  In total, we identified 219 sources in 6\,cm image, see Fig.~\ref{f:coresoverallvla}. The `isophotal flux' ($S_{\rm 6cm}$) are calculated, which is the flux of the area that has flux density all above the threshold in SExtractor. Defining the radius of a core as $R = \sqrt{ab}$, we plot $S_{\rm 6cm}$ versus $R$ of the 219 sources in Fig.~\ref{f:r-flx}. As shown in Fig.~\ref{f:r-flx}, most of the sources have $R$ between 5 and 30\,pc, and $S_{\rm 6cm}$ between 0.1 and 300\,mJy. The optical depth of free-free emission can be estimated as:
  \begin{equation}
    \tau = 441 
    \left(\frac{\nu}{\rm GHz}\right)^{-2}
    \left(\frac{T_{\rm e}}{\rm K}\right)^{-1} 
    \left(\frac{\theta_{a}\theta_{b}}{\rm arcsec^2}\right)^{-1}
    \frac{S_\nu}{\rm mJy},
  \end{equation}
  in which $\nu$ is the frequency, here we use 6\,GHz. $T_{\rm e}$ is the electron temperature, based on the typical values in \hii regions \cite[see e.g.][]{Mehringer:1993aa}, we use $10^4$\,K. The angular counterparts of $a$ and $b$ are denoted as $\theta_{a}$ and $\theta_{b}$, respectively. From Fig.~\ref{f:r-flx}, we can see that all the cores have $\tau<10^{-2}$ and vast majority have $\tau<10^{-3}$. Therefore, we assume that the free-free emission from these 219 cores are all optically thin. Under the optically thin assumption, we calculate the  emission measure (EM) of all the sources as:
  \begin{equation}
    {\rm EM} = 5.36\times10^3 
    \left(\frac{\nu}{\rm GHz}\right)^{0.1} 
    \left(\frac{T_{\rm e}}{\rm K}\right)^{0.35} 
    \left(\frac{\theta_{a}\theta_{b}}{\rm arcsec^2}\right)^{-1}
    \frac{S_{\rm 6cm}}{ \rm mJy}.
  \end{equation}
  As shown in Fig.~\ref{f:r-flx}, most of the sources have EM ranging from $10^6$ to $10^8$\,pc\,cm$^{-6}$. About half of the sources have EM larger than $10^7$\,pc\,cm$^{-6}$. Knowing EM and $a$, $b$, we derive the volume density of electrons $n = \sqrt{{\rm EM} / ab}$. Most of the sources have $n$ around $10^4$\,cm$^{-3}$. 

  According to the criterion of UCH{\sc ii} regions \citep[see e.g.][]{Kurtz:2005aa}, UCH{\sc ii} regions should have EM$\gtrsim10^7$\,pc\,cm$^{-6}$, $n\gtrsim10^4$\,cm$^{-3}$, and $R\lesssim0.1$\,pc. Thus, we select the sources with EM$\gtrsim10^7$, which also fulfills the later two criteria, as UCH{\sc ii} regions (see blue points in Fig.~\ref{f:r-flx}). In total, we identified 101 UCH{\sc ii} regions, which are plotted as white ellipses in Fig.~\ref{f:coresoverallvla}. The spatial distribution of the UCH{\sc ii} regions is not homogeneous. In M and N, 48 and 38 UCH{\sc ii} regions are identified, respectively. While in S we identified only two UCH{\sc ii} regions and in DS none. 

  We also compare the mass of ionized gas ($M_\mathrm{H{ii}}$) of all the sources with that of the UCH{\sc ii}regions. The $M_\mathrm{H{ii}}$ can be calculated as:
  \begin{equation}
  \begin{aligned}
    {M_\mathrm{H{ii}}} =& 5.15\times10^{-12} 
    \left(\frac{\nu}{\rm GHz}\right)^{0.5} 
    \left(\frac{T_{\rm e}}{\rm K}\right)^{0.175} 
    \left(\frac{D}{\rm pc}\right)^{2.5} \\
    &\times\left(\frac{\theta_{a}\theta_{b}}{\rm arcsec^2}\right)^{0.75}
    \left(\frac{S_{\rm 6cm}}{ \rm mJy}\right)^{0.5}.
    \end{aligned}
  \end{equation}
  As shown in Fig.~\ref{f:mhii}, the 219 sources have $M_\mathrm{H{ii}}$ ranging from $\sim 4\times10^{-3}\,M_{\odot}$ to $\sim 2\,M_{\odot}$. The 101 UCH{\sc ii} regions have $M_\mathrm{H{ii}}$ ranging from $\sim 1\times10^{-2}\,M_{\odot}$ to $\sim 2\,M_{\odot}$. The average $M_\mathrm{H{ii}}$ of the 101 UCH{\sc ii} regions ($0.4\,M_{\odot}$) is twice of that of all the 219 cores ($0.2\,M_{\odot}$). 


  \begin{figure*}[ht]
          \begin{center}
          \includegraphics[width=0.85\textwidth]{./plot/overall_cores_006.pdf}
          \caption[Cores in 6\,cm image]{Sources and UC\hii regions identified in 6\,cm image, which are marked as blue and white ellipses, respectively, corresponding to the elliptical parameters fitted by SExtractor. Light gray ellipses indicates the cores identified in the 3\,mm image. Notable regions are marked.
           Beam sizes are marked on the lower left corner of each panel, in format $(\theta_{\rm maj}, \theta_{\rm min}, {\rm PA})$. 
           }
          \label{f:coresoverallvla}
          \end{center}
  \end{figure*}

  \begin{figure*}[ht]
          \begin{center}
          \includegraphics[width=0.7\textwidth]{./plot/r-flx.pdf}
          \caption[Relationship between flux and R of 6cm sources.]{Relationship between $S_{\rm 6cm}$ and $R$ of 6cm sources. Red lines indicates various EM. Light blue dashed lines indicates different $n$. Gray dashed lines indicates $\tau$.
           }
          \label{f:r-flx}
          \end{center}
  \end{figure*}

  \begin{figure}[ht]
          \begin{center}
          \includegraphics[width=0.4\textwidth]{./plot/mhii.pdf}
          \caption[Distribution of $M_\mathrm{H{ii}}$]{Distribution of mass of ionized gas ($M_\mathrm{H{ii}}$) of all the sources (gray) and UC\hii regions (blue). Average value of each distribution are marked in correspondent colors.
           }
          \label{f:mhii}
          \end{center}
  \end{figure}

 \subsection{Dust Cores} % (fold)
 \label{sub:dust_cores}
  \begin{figure*}[ht]
          \begin{center}
          \includegraphics[width=0.85\textwidth]{./plot/overall_cores_096.pdf}
          \caption[Cores in 6\,cm image]{3mm cores.
           }
          \label{f:coresoverallalma}
          \end{center}
  \end{figure*}


  \begin{figure*}[ht]
          \begin{center}
          \includegraphics[width=0.48\textwidth]{./plot/006-096-flx.pdf}
          \includegraphics[width=0.48\textwidth]{./plot/006-096-flx_2.pdf}
          \caption[Relationship between flux and R of 6cm sources.]{Flux comparison between 6\,cm and 3\,mm
           }
          \label{f:vla-alma}
          \end{center}
  \end{figure*}

 
 % subsection dust_cores (end)

  % subsection cores_in_6_cm (end)



  
  % section results (end)

% section introduction (end)
\bibliographystyle{aa} % style aa.bst
\bibliography{ref}

\begin{appendix}

\section{RMS maps} % (fold)
\label{sec:rms_maps}

  In this section we introduce the production of the RMS maps of the 6\,cm and 3\,mm data used in this paper.  Firstly, we masked out the image at where the signal is above a certain threshold to create a `masked image'. In this paper, the threshold for both of 6\,cm and 3\,mm images were 1\,mJy/beam. In SciPy \citep{2020SciPy-NMeth}, we applied the algorithm \texttt{ndimage.generic\_filter}, to the masked image. The filter function was defined as a standard deviation function. A parameter called \texttt{size} controls the subregion of the masked image that is given as input to the filter algorithm each time. For both of 6\,cm and 3\,mm images, \texttt{size} was set to 100\,pixels, which corresponds to 5\arcsec. Thus, an masked RMS map was generated, in which where the signal was above the threshold is empty.  To fill the masked regions, we used cubic interpolation over the whole image. In the interpolation, patches of the size of $100\times100$\,pixels (5\arcsec$\times$5\arcsec) were treated as one data point, which reduces the complexity of the computation. Thus we obtain rms maps as shown in Fig.~\ref{f:rmsmap_a} and \ref{f:rmsmap_b}


  

  \begin{figure}[ht]
          \begin{center}
          \includegraphics[width=0.45\textwidth]{./plot/abcd_006_rms.pdf}
          % \includegraphics[width=0.45\textwidth]{./plot/096_rms.pdf}\\     
          \caption[RMS map of 6\,cm image.]{RMS map of 6\,cm  image. The image has two profile cuts, in blue and red. The offset of each profile increase from left to right and from bottom to top. The profile is below each image.
           }
          \label{f:rmsmap_a}
          \end{center}
  \end{figure}

  \begin{figure}[ht]
          \begin{center}
          % \includegraphics[width=0.45\textwidth]{./plot/abcd_006_rms.pdf}
          \includegraphics[width=0.45\textwidth]{./plot/096_rms.pdf}     
          \caption[RMS map of  3\,mm image.]{RMS map of 3\,mm image. The image has two profile cuts, in blue and red. The offset of each profile increase from left to right and from bottom to top. The profile is below each image.
           }
          \label{f:rmsmap_b}
          \end{center}
  \end{figure}


\section{Core identification} % (fold)
\label{sec:core_identification}

% section core_identification (end)
% section rms_maps (end)
\end{appendix}

\end{document}










